\documentclass[a4paper,12pt]{jarticle}
\usepackage[dvipdfmx]{graphicx}
\usepackage{amsmath}
\usepackage{subfigure}
\usepackage{comment}

\setlength{\hoffset}{0cm}
\setlength{\oddsidemargin}{-3mm}
\setlength{\evensidemargin}{-3cm}
\setlength{\marginparsep}{0cm}
\setlength{\marginparwidth}{0cm}
\setlength{\textheight}{24.7cm}
\setlength{\textwidth}{17cm}
\setlength{\topmargin}{-45pt}

\renewcommand{\baselinestretch}{1.6}
\renewcommand{\floatpagefraction}{1}
\renewcommand{\topfraction}{1}
\renewcommand{\bottomfraction}{1}
\renewcommand{\textfraction}{0}
\renewcommand{\labelenumi}{(\arabic{enumi})}
%\renewcommand{\figurename}{Fig.} %図をFig.にする

\begin{comment}
%図のキャプションからコロン:を消す
\makeatletter
\long\def\@makecaption#1#2{% #1=図表番号、#2=キャプション本文
\sbox\@tempboxa{#1. #2}
\ifdim \wd\@tempboxa >\hsize
#1 #2\par 
\else
\hb@xt@\hsize{\hfil\box\@tempboxa\hfil}
\fi}
\makeatother
% 
\end{comment}

\title{電機システム制御特論 \ レポート課題\\
Exercise\\
}
\author{\vspace{40mm}\\
九州工業大学大学院 \hspace{0mm} 工学府\\
機械知能工学専攻\ \hspace{0mm} 知能制御工学コース \\
\vspace{5mm}\\
所属:\ 西田研究室\\
学籍番号:\ 16344217\\
提出者氏名:\ 津上 \hspace{0mm} 祐典\\\vspace{5mm}\\ }
\date{平成28年\ 4月\ 15日}

\begin{document}

%表紙
\titlepage
\maketitle
\thispagestyle{empty}

\newpage

\section*{問題}
Consider the actuator is made by stacking 100 PZT plates eith the source
voltage is 200 V. Each plate has a thickness of 120$\mu$m, a length of
10mm, and a width of 15mm. Compute the generated force and 
displacement. Assume that Young's modulus of the material is
$5.3×10^{10}$(N/$\rm{m}^2$).

\section*{解答}

アクチュエータの変形量$\Delta h$は
\begin{equation}
 \Delta h = Nd_{33}V = 100×400×10^{-12}×200 = 8[\mu \rm{m}]
\end{equation}
となる.また,アクチュエータの面積$A$,高さ$H$は,それぞれ
\begin{eqnarray}
 \begin{cases}
  A = 10 × 10^{-3} × 15 × 10^{-3} = 1.5 × 10^{-4}  [$\rm{m}^2$] \\
  H = 120 × 10^{-6} × 100 = 0.012 [\rm{m}] &
 \end{cases}
\end{eqnarray}
となる.したがって発生する力$F_{33}$は,ヤング率を$Y$とすると
\begin{equation}
  F_{33} = YA\Delta h/H = 5.3 × 10^{10} × 1.5 ×10^{-4} × 8 ×
   10^{-6} / 0.12 = 5300 \rm{(N)}
\end{equation}
となる.

\end{document}
