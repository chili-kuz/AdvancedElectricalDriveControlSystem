\documentclass[a4paper,12pt]{jarticle}
\usepackage[dvipdfmx]{graphicx}
\usepackage{amsmath}
\usepackage{subfigure}
\usepackage{comment}

\setlength{\hoffset}{0cm}
\setlength{\oddsidemargin}{-3mm}
\setlength{\evensidemargin}{-3cm}
\setlength{\marginparsep}{0cm}
\setlength{\marginparwidth}{0cm}
\setlength{\textheight}{24.7cm}
\setlength{\textwidth}{17cm}
\setlength{\topmargin}{-45pt}

\renewcommand{\baselinestretch}{1.6}
\renewcommand{\floatpagefraction}{1}
\renewcommand{\topfraction}{1}
\renewcommand{\bottomfraction}{1}
\renewcommand{\textfraction}{0}
\renewcommand{\labelenumi}{(\arabic{enumi})}
%\renewcommand{\figurename}{Fig.} %図をFig.にする


%図のキャプションからコロン:を消す
\makeatletter
\long\def\@makecaption#1#2{% #1=図表番号、#2=キャプション本文
\sbox\@tempboxa{#1. #2}
\ifdim \wd\@tempboxa >\hsize
#1 #2\par 
\else
\hb@xt@\hsize{\hfil\box\@tempboxa\hfil}
\fi}
\makeatother
% 


\title{電機システム制御特論 \\
Assignment (2016/05/13)\\
}
\author{\vspace{40mm}\\
九州工業大学大学院 \hspace{0mm} 工学府\\
機械知能工学専攻\ \hspace{0mm} 知能制御工学コース \\
\vspace{5mm}\\
所属:\ 西田研究室\\
学籍番号:\ 16344217\\
提出者氏名:\ 津上 \hspace{0mm} 祐典\\\vspace{5mm}\\ }
\date{平成28年\ 5月\ 20日}

\begin{document}

%表紙
\titlepage
\maketitle
\thispagestyle{empty}

\newpage

%%%%%%%%%%%%%%%%%%%%%%%%
\section*{問題}
%%%%%%%%%%%%%%%%%%%%%%%%
Design a DC speed control system, and show demonstrations for the four quadrant operation.
%
\begin{table}[h]
 \centering
 \caption{DC motor dimension}
 \label{table:DC_dim}
 \begin{tabular}{c|c|c} \hline
  名前& 記号& 数値\\\hline
  Rated Power [kW]              &$P$&150  \\\hline
  Rated Voltage [V]             &$V$&450  \\\hline
  Armature resistance [$\Omega$]&$R_a$& 0.15 \\\hline
  Armature inductance [H]       &$L_a$&0.003\\\hline
  Moment of inerria [kg$\rm {m^3}$] &$J$&150  \\\hline
  Emf constant [Vs/rad]       &$K_E$&8.50 \\\hline
  Base speed [rpm]            &$\omega$&500  \\\hline
 \end{tabular}
\end{table}
%
%%%%%%%%%%%%%%%%%%%%%%%%
\section{}
%%%%%%%%%%%%%%%%%%%%%%%%
はじめにDCモータのブロック線図を時に示す.

\end{document}
